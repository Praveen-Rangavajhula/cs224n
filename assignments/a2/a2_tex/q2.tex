\titledquestion{Machine Learning \& Neural Networks}[8] 
\begin{parts}

    
    \part[4] Adam Optimizer\newline
        Recall the standard Stochastic Gradient Descent update rule:
        \alns{
            	\btheta_{t+1} &\gets \btheta_t - \alpha \nabla_{\btheta_t} J_{\text{minibatch}}(\btheta_t)
        }
        where $t+1$ is the current timestep, $\btheta$ is a vector containing all of the model parameters, ($\btheta_t$ is the model parameter at time step $t$, and $\btheta_{t+1}$ is the model parameter at time step $t+1$), $J$ is the loss function, $\nabla_{\btheta} J_{\text{minibatch}}(\btheta)$ is the gradient of the loss function with respect to the parameters on a minibatch of data, and $\alpha$ is the learning rate.
        Adam Optimization\footnote{Kingma and Ba, 2015, \url{https://arxiv.org/pdf/1412.6980.pdf}} uses a more sophisticated update rule with two additional steps.\footnote{The actual Adam update uses a few additional tricks that are less important, but we won't worry about them here. If you want to learn more about it, you can take a look at: \url{http://cs231n.github.io/neural-networks-3/\#sgd}}
            
        \begin{subparts}

            \subpart[2]First, Adam uses a trick called {\it momentum} by keeping track of $\bm$, a rolling average of the gradients:
                \alns{
                	\bm_{t+1} &\gets \beta_1\bm_{t} + (1 - \beta_1)\nabla_{\btheta_t} J_{\text{minibatch}}(\btheta_t) \\
                	\btheta_{t+1} &\gets \btheta_t - \alpha \bm_{t+1}
                }
                where $\beta_1$ is a hyperparameter between 0 and 1 (often set to  0.9). \textbf{Briefly explain in 2--4 sentences} (you don't need to prove mathematically, just give an intuition) how using $\bm$ stops the updates from varying as much and why this low variance may be helpful to learning, overall.\newline

                \ifans{

                    Using $\bm$ stops updates from varying too much because it maintains a
                    rolling average of past gradients, effectively smoothing out fluctuations
                    in gradient directions. This reduces noise from sudden gradient changes, ensuring that updates
                    are more consistent rather than erratic.
                    \\
                    As a result, the optimization process prioritizes movement in directions that
                    consistently lead toward the minimum, while variations in other directions tend to cancel out.
                    This lower variance helps the model converge faster and more efficiently, making updates
                    to the weights (or word vectors) more stable, ultimately improving learning and accuracy.

                }

                
            \subpart[2] Adam extends the idea of {\it momentum} with the trick of {\it adaptive learning rates} by keeping track of  $\bv$, a rolling average of the magnitudes of the gradients:
                \alns{
                	\bm_{t+1} &\gets \beta_1\bm_{t} + (1 - \beta_1)\nabla_{\btheta_t} J_{\text{minibatch}}(\btheta_t) \\
                	\bv_{t+1} &\gets \beta_2\bv_{t} + (1 - \beta_2) (\nabla_{\btheta_t} J_{\text{minibatch}}(\btheta_t) \odot \nabla_{\btheta_t} J_{\text{minibatch}}(\btheta_t)) \\
                	\btheta_{t+1} &\gets \btheta_t - \alpha \bm_{t+1} / \sqrt{\bv_{t+1}}
                }
                where $\odot$ and $/$ denote elementwise multiplication and division (so $\bz \odot \bz$ is elementwise squaring) and $\beta_2$ is a hyperparameter between 0 and 1 (often set to  0.99). Since Adam divides the update by $\sqrt{\bv}$, which of the model parameters will get larger updates?  Why might this help with learning? \textbf{Briefly explain in 2--4 sentences}.
                
                \ifans{

                    $\sqrt{v}$ tracks the magnitude of a parameter’s gradient over multiple iterations by maintaining
                    a rolling average of squared gradients. If a parameter’s gradient magnitude remains small over time,
                    dividing by $\sqrt{v}$ results in larger updates, ensuring that parameters associated with rare
                    words receive sufficient training. This helps improve generalization by allowing the model
                    to learn better embeddings for infrequent words that might otherwise be poorly trained.

                    \hrulefill

                    \small
                    \textbf{(Extra Intuition - Not Part of Required Answer)}

                    Conversely, parameters with larger gradients receive smaller updates, preventing rapid
                    fluctuations and ensuring smoother learning. This reduces oscillations, avoids overshooting the
                    minimum, and leads to more stable convergence. By balancing these adjustments, Adam helps both rare
                    feature learning and overall optimization stability.

                    \normalsize
                }

                \end{subparts}
        
        
            \part[4] 
            Dropout\footnote{Srivastava et al., 2014, \url{https://www.cs.toronto.edu/~hinton/absps/JMLRdropout.pdf}} is a regularization technique. During training, dropout randomly sets units in the hidden layer $\bh$ to zero with probability $p_{\text{drop}}$ (dropping different units each minibatch), and then multiplies $\bh$ by a constant $\gamma$. We can write this as:
                \alns{
                	\bh_{\text{drop}} = \gamma \bd \odot \bh
                }
                where $\bd \in \{0, 1\}^{D_h}$ ($D_h$ is the size of $\bh$)
                is a mask vector where each entry is 0 with probability $p_{\text{drop}}$ and 1 with probability $(1 - p_{\text{drop}})$. $\gamma$ is chosen such that the expected value of $\bh_{\text{drop}}$ is $\bh$:
                \alns{
                	\mathbb{E}_{p_{\text{drop}}}[\bh_\text{drop}]_i = h_i \text{\phantom{aaaa}}
                }
                for all $i \in \{1,\dots,D_h\}$. 
            \begin{subparts}
            \subpart[2]
                What must $\gamma$ equal in terms of $p_{\text{drop}}$? Briefly justify your answer or show your math derivation using the equations given above.

            \ifans{

                We are given:

                \[
                \bh_{\text{drop}} = \gamma \bd \odot \bh
                \]

                where:
                \begin{itemize}
                    \item $\bd \in \{0,1\}^{D_h}$ is a mask vector where each element is $0$ with probability $p_{\text{drop}}$ and $1$ with probability $1 - p_{\text{drop}}$.
                    \item $\gamma$ is a scaling factor ensuring that the expected value of $\bh_{\text{drop}}$ remains equal to $\bh$.
                \end{itemize}

                Thus, we need to satisfy:

                \[
                \mathbb{E}_{p_{\text{drop}}}[\bh_{\text{drop}}]_i = h_i.
                \]

                Expanding $\bh_{\text{drop}}$:

                \[
                \mathbb{E}_{p_{\text{drop}}}[\bh_{\text{drop}}]_i = \mathbb{E}_{p_{\text{drop}}}[\gamma d_i h_i].
                \]

                Since $\gamma$ and $h_i$ are constants with respect to the expectation, we can factor them out:

                \[
                \mathbb{E}_{p_{\text{drop}}}[\bh_{\text{drop}}]_i = \gamma h_i \mathbb{E}_{p_{\text{drop}}}[d_i].
                \]

                Since $d_i$ takes the value $1$ with probability $1 - p_{\text{drop}}$ and $0$ with probability $p_{\text{drop}}$, its expectation is:

                \[
                \mathbb{E}[d_i] = 1 \cdot (1 - p_{\text{drop}}) + 0 \cdot (p_{\text{drop}}) = 1 - p_{\text{drop}}.
                \]

                Substituting this back:

                \[
                \mathbb{E}_{p_{\text{drop}}}[\bh_{\text{drop}}]_i = \gamma h_i (1 - p_{\text{drop}}).
                \]

                Setting this equal to $h_i$:

                \[
                \gamma h_i (1 - p_{\text{drop}}) = h_i.
                \]

                Dividing both sides by $h_i$ (assuming $h_i \neq 0$):

                \[
                \gamma = \frac{1}{1 - p_{\text{drop}}}.
                \]

                Thus, the required scaling factor is:

                \[
                \gamma = \frac{1}{1 - p_{\text{drop}}}.
                \]

            }
            
          \subpart[2] Why should dropout be applied during training? Why should dropout \textbf{NOT} be applied during evaluation? \textbf{Briefly explain in 2--4 sentences}. \textbf{Hint:} it may help to look at the dropout paper linked. \newline

            \ifans{

                Dropout is applied during training because it acts as a regularization technique, reducing overfitting
                by preventing the network from over-relying on specific neurons. By randomly dropping units during each
                iteration, the model is forced to learn redundant representations, improving its ability to generalize
                to unseen data. This helps reduce variance and prevents the network from memorizing patterns that are
                too specific to the training set.

                Dropout should \textbf{not} be used during evaluation because it would remove some of the learned
                connections, effectively weakening the model. Instead, during evaluation, we use the entire network
                and scale the weights appropriately to match the expected activations during training. This ensures
                that the model can fully utilize what it has learned without unnecessary disruptions.

            }
         
        \end{subparts}


\end{parts}
